\documentclass[11pt]{article}
\usepackage[utf8]{inputenc}
\usepackage{amsmath}
\usepackage{amssymb}
\usepackage{tikz-cd}
\usepackage[margin = 1in]{geometry}

\usepackage{titlesec}
\titleformat{\section}
  {\normalfont\scshape}{\thesection}{1em}{}

\usepackage{authblk}

\title{Computing the Geometric Category of Finite Topological Spaces}
\author[1]{\textbf{Shelley Kandola}}
\author[2,1]{Elizabeth Munch}
\affil[1]{Dept.~of Mathematics}
\affil[2]{Dept.~of Computational Math, Science, and Engineering}
\affil[\phatom{}]{Michigan State University}

\date{}

\begin{document}

\maketitle
\vspace{-.5in} %Getting rid of the date space, too lazy to implement titling
\section{Introduction}
The topological complexity of a space is the minimal number of continuous motion planning rules required to move from one point in the space to another.
Typically, the space will represent the space of configurations of some mechanical system, such as a robot.
In this way, topological complexity is useful for determining algorithms for robot motion planning problems.
Restricting our spaces to finite topological spaces might allow for more accurate models of real-life mechanical systems.
In this talk, we present algorithms for computing the geometric category of topological spaces, providing an upper-bound for the topological complexity.

\section{Topological Complexity}

Topological Complexity was first described by Michael Farber in \cite{Farber2001}.
Informally, the Topological Complexity (TC) of a path-connected space $X$ is the minimal number of continuous motion planning rules required to navigate between any two points in $X$. Typically, $X$ represents the space of configurations of some mechanical system, such as a robot.
More formally, consider the following commutative diagram:

\begin{center}
   \begin{tikzcd}
X \arrow[rd,"\Delta"] \arrow[r, "c_x"] & X^{[0,1]}\arrow[d,"\pi"] \\
& X\times X
\end{tikzcd}
\end{center}


Here, $X$ is our path-connected space of configurations.
The map $c_x$ sends a point $x\in X$ to the constant path at $x$ in $X^{[0,1]}$.
The diagonal map $\Delta$ sends $x$ to $(x,x)\in X \times X$, and the projection map $\pi$ is the fibrant replacement of $\Delta$.
It takes a path $\gamma \in X^{[0,1]}$, and sends it to its start and end points $(\gamma(0),\gamma(1)) \in X \times X$.

The {\bf Schwarz genus} $\mathfrak{g}(p)$ of a fibration $p:E \to B$ is the minimal number $k$ such that there exists an open covering $U_1,\hdots,U_k$ of $B$ where each set $U_i$ admits a local $p$-section (see \cite{Schwarz1958}).
That is, each $U_i$ has an associated map $s_i$ such that $p \circ s_i \simeq 1_B$. The \textbf{topological complexity} of $X$ is given by $\mathfrak{g}(\pi)$, and it is homotopy-invariant.

Unlike other topological invariants such as homology, there is no algorithm for computing the TC of a space.
There are a few spaces whose TC has been manually determined.
For the most part, TC is calculated by determining lower- and upper-bounds.
The upper-bound of interest to us is the Lusternik-Schnirelmann category of \cite{LS34}.
The \textbf{Lusternik-Schnirelmann category} of a space $X$, denoted ${LS}(X)$, is the minimal number $k$ such that $X$ can be covered in $k$ open sets $U_i \subseteq X$ whose inclusion maps $\iota_i: U_i \hookrightarrow X$ are nullhomotopic.
Farber proves in \cite{Farber2001} that $TC(X) \leq LS(X\times X)$.
An obvious upper-bound for the Lusternik-Schnirelmann category is the minimal number of contractible open sets covering a space; in \cite{Fernandez-Ternero2018}, they describe this as the \textbf{geometric category} of a space, denoted $\text{gcat}(X)$.

Realistically, a programmer might not be interested in designing a robot that can be in uncountably many positions.
This motivates the application of topological complexity to finite topological spaces.
When finite spaces are $T_0$, they yield a partial order, $\leq$. Given two points $x$ and $y$ in a finite space $X$, we say that $x \leq y$ if the smallest open neighborhood containing $x$ is a subset of the smallest open neighborhood containing $y$. This partial order allows us to represent finite spaces as Hasse diagrams, which are directed acyclic graphs. A point in a finite space is called \textbf{maximal} if its in-degree in the Hasse diagram is 0.

An adaptation of TC for finite topological spaces was described in \cite{Tanaka2018}, called Combinatorial Complexity (CC).
In that same paper, the author proved that for a finite topological space $X$, $CC(X) = TC(X)$, and so all the traditional tools of TC can be applied.
Tanaka showed that if $m$ is the number of maximal elements in a finite space, we have the inequality $TC(X) \leq m^2$.
Combining this bound the the others mentioned above, we have $TC(X) \leq LS(X\times X) \leq gcat(X \times X) \leq m^2$ for any finite path-connected space $X$.
To date, there are no known finite spaces $X$ for which $TC(X) < LS(X \times X)$.
In particular, these categorical open sets are either contractible open sets, or disjoint unions of contractible sets.


\section{Algorithms and Results}
Because of the combinatorial nature of finite topological spaces, we are able to manually construct all possible contractible open subsets covering a space.
In this work, we will discuss the creation of an algorithm for determining the geometric category of a finite space.
By computing this value, we can provide an upper-bound for the TC of finite spaces that is tighter than $m^2$.
This is done by combining the open set representations with graph theoretic operations on the Hasse diagram.
In the future, we would like to investigate the complexity of this computation, and use our related open-source code to search for a space which can show that the bound $TC(X) \leq LS(X \times X)$ is not tight.


\bibliography{abstract}{}
\bibliographystyle{alpha}

\end{document}
